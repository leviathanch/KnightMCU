Here's a block diagram that outlines the various modules required for implementing the features we discussed earlier:

\begin{center}
\begin{tikzpicture}[node distance=2cm, every node/.style={align=center},scale=0.5, every node/.style={scale=0.5}]
% Nodes
\node [block] (matrixmul) {Matrix Multiplication};
\node [block, right of=matrixmul, xshift=2.5cm] (convolution) {Convolution};
\node [block, right of=convolution, xshift=2.5cm] (activation) {Activation Functions};
\node [block, below of=activation] (pooling) {Pooling};
\node [block, left of=pooling, xshift=-2.5cm] (elementwise) {Element-wise Operations};
\node [block, left of=elementwise, xshift=-2.5cm] (reduction) {Reduction Operations};
\node [block, below of=reduction] (normalization) {Normalization};
\node [block, below of=normalization] (softmax) {Softmax};
\node [block, below of=softmax] (loss) {Loss Functions};
\node [block, left of=loss, xshift=-2.5cm] (regularization) {Regularization};
\node [block, above of=matrixmul, xshift=-2.5cm] (padding) {Padding};
% Arrows
\draw [->] (matrixmul) -- (convolution);
\draw [->] (convolution) -- (activation);
\draw [->] (activation) -- (pooling);
\draw [->] (pooling) -- (elementwise);
\draw [->] (elementwise) -- (reduction);
\draw [->] (reduction) -- (normalization);
\draw [->] (normalization) -- (softmax);
\draw [->] (softmax) -- (loss);
\draw [->] (loss) -- (regularization);
\draw [->] (regularization) -- (padding);
\draw [->] (padding) -- (convolution);
\end{tikzpicture}
\end{center}

In this diagram, each block represents a module responsible for a specific functionality. Here's a detailed description of each block:

\begin{enumerate}
\item{Matrix Multiplication: This module performs matrix multiplication operations, which are essential for neural network computations.}
\item{Convolution: The convolution module handles the convolution operations commonly used in convolutional neural networks (CNNs) for tasks like image recognition and natural language processing.}
\item{Activation Functions: This module implements activation functions like ReLU, sigmoid, and tanh, introducing non-linearity into the neural network model.}
\item{Pooling: The pooling module performs operations like max pooling or average pooling, which downsample feature maps in CNNs, reducing their spatial dimensions.}
\item{Element-wise Operations: This module handles element-wise operations such as element-wise addition and multiplication, performed on corresponding elements of tensors.}
\item{Reduction Operations: The reduction module implements operations like sum, mean, and max, used to aggregate values across tensor dimensions.}
\item{Normalization: This module handles normalization techniques like batch normalization or layer normalization, which normalize activations within a neural network layer.}
\item{Softmax: The softmax module converts a vector of real numbers into a probability distribution, commonly used for classification tasks.}
\item{Loss Functions: This module implements loss functions like cross-entropy or mean squared error, measuring the difference between predicted and actual values during training.}
\item{Regularization: The regularization module handles techniques like L1 or L2 regularization, which prevent overfitting by adding penalty terms to the loss function.}
\item{Padding: This module takes care of padding operations, adding extra elements to tensors, typically at the borders, to preserve spatial dimensions during convolution operations.}
\end{enumerate}